%Premodel Workflow
%8
\subsection{Introduction}
This chapter discusses setting data, preparing data, and premodel dimensionality reduction.
These are not the attractive parts of machine learning (ML), but they often turn out to be
what determines if a model will work or not.

There are three main parts to the chapter. 

Firstly, we'll create fake data; this might seem
trivial, but creating fake data and fitting models to fake data is an important step in model
testing. It's more useful in situations where we implement an algorithm from scratch, but I'll
cover it here for completeness, and in the event you don't have data of your own, you can just
create it. 

Secondly, we'll look at broadly handling data transformations as a preprocessing
step, which includes data imputation, categorical variable encoding, and so on. 

Thirdly,
we'll look at situations where we have a large number of features relative to the number of
observations we have.
%=================================================%
This chapter, especially the first half, will set the stage for the later chapters. In order to use
scikit-learn, data is required. The first two sections will discuss acquiring the data; the rest of
the first half will discuss preparing this data for use.

This book is written using scikit-learn 0.15, NumPy 1.9, and pandas 0.13.
There are other packages used as well, so it's advisable that you refer to
the installation instructions included in this book.
%=================================================%
