\documentclass[SKL-MASTER.tex]{subfiles}
\begin{document}
\Large
\section*{Getting sample data from external sources}
\begin{itemize}
\item If possible, try working with a familiar dataset while working through this book; in order to level
the field, built-in datasets will be used. 
\item The built-in datasets can be used as stand-ins to test
several different modeling techniques such as regression and classification.
\item  These are, for the
most part, famous datasets. This is very useful as papers in various fields will often use these
datasets for authors to put forth how their model fits as compared to other models.

\item I recommend you use IPython to run these commands as they are
presented. Muscle memory is important, and it's best to get to the
point where basic commands take no extra mental effort. \item An even
better way might be to run IPython Notebook. If you do, make sure
to use the \texttt{\%matplotlib inline} command; this will allow you
to see the plots in Notebook.
\end{itemize}
%===========================================================================================%
%-- Chapter 1

\subsection*{Preparation}
The datasets in scikit-learn are contained within the datasets module. Use the following
command to import these datasets:
%------------------------%
\begin{framed}
\begin{verbatim}
>>> from sklearn import datasets
>>> import numpy as np
\end{verbatim}
\end{framed}
%------------------------ %
From within IPython, run \texttt{datasets.$\ast$?}, which will list everything available within the
datasets module.
\section*{Working with Data Sets}
There are two main types of data within the datasets module. 
\begin{enumerate}
\item Smaller test datasets
are included in the sklearn package and can be viewed by running \texttt{datasets.load}$\_\ast$?.

\item Larger datasets are also available for download as required. The latter are not included in
sklearn by default; however, at times, they are better to test models and algorithms due
to sufficient complexity to represent realistic situations.
\end{enumerate}

Datasets are included with sklearn by default; to view these datasets, run datasets.
%------------------------%
\begin{framed}
\begin{verbatim}
load_*?. 
\end{verbatim}
\end{framed}
%------------------------ %
There are other types of datasets that must be fetched. These datasets are
larger, and therefore, they do not come within the package. This said, they are often
better to test algorithms that might be used in the wild.
First, load the boston dataset and examine it:
%------------------------%
\begin{framed}
\begin{verbatim}
>>> boston = datasets.load_boston()
>>> print boston.DESCR #output omitted due to length
\end{verbatim}
\end{framed}
%------------------------ %
\texttt{DESCR} will present a basic overview of the data to give you some context.
Next, fetch a dataset:
%------------------------%
\begin{framed}
	\begin{verbatim}
>>> housing = datasets.fetch_california_housing()
downloading Cal. housing from http://lib.stat.cmu.edu [...]
>>> print housing.DESCR #output omitted due to length
\end{verbatim}
\end{framed}
%------------------------ %
\subsection*{Bunch Objects} % % - How it works…
\begin{itemize}
\item When these datasets are loaded, they aren't loaded as NumPy arrays. They are of type \texttt{Bunch}.
\item A \texttt{Bunch} is a common data structure in Python. \item It's essentially a dictionary with the keys
added to the object as attributes.
\end{itemize}

%===========================================================================================%
%Premodel Workflow
%10
To access the data using the data attribute, which is a NumPy array containing the
independent variables, the target attribute has the dependent variable:

%------------------------ %
\begin{framed}
\begin{verbatim}
>>> X, y = boston.data, boston.target
\end{verbatim}
\end{framed}
\begin{itemize}
\item There are various implementations available on the Web for the Bunch object; it's not too
difficult to write on your own. scikit-learn defines Bunch (as of this writing) in the base module.
\item It's available in GitHub at https://github.com/scikit-learn/scikit-learn/blob/
master/sklearn/datasets/base.py.
\end{itemize}

%====================================================================== %
%There's more…

When you fetch a dataset from an external source it will, by default, place the data in your
home directory under \texttt{scikit\_learn\_data/}; this behavior is configurable in two ways:
\begin{enumerate}
\item To modify the default behavior, set the \texttt{SCIKIT\_LEARN\_DATA} environment variable
to point to the desired folder.
\item The first argument of the fetch methods is \texttt{data\_home}, which will specify the home
folder on a case-by-case basis.
\end{enumerate}
It is easy to check the default location by calling \texttt{datasets.get\_data\_home()}.
See also

\subsection*{UCI Machine Learning Repository}
The UCI Machine Learning Repository is a great place to find sample datasets. Many of the
datasets in scikit-learn are hosted here; however, there are more datasets available. Other
notable sources include KDD, your local government agency, and Kaggle competitions.



\end{document}
